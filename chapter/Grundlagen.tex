%!TEX root = ../Thesis.tex
\section{Grundlagen}

\subsection{Schrift}
\label{sec:schrift}

\subsubsection{Schriftgrößen}
\label{sec:schriftgroessen}
\tiny Das ist sehr kleine Schrift\\
\small Das ist kleine Schrift\\
\normalsize Das ist normale Schrift\\
\large Das ist große Schrift\\
\Large Das ist größere Schrift\\
\LARGE Das ist noch größere Schrift\\
\huge Das ist riesige Schrift\\
\Huge Das ist noch riesigere Schrift\\
\scriptsize Das ist Script Schrift\\
\footnotesize Das ist Fußnoten Schrift
\normalsize

\subsubsection{Schrift Typen}
\label{sec:Schrift Typen}
\textbf{Das ist ein fetter Text}\\
\textit{Das ist ein kursiver Text}\\
\underline{Das ist ein unterstrichener Text}\\
\textsc{Das ist ein kapitälchen Text}\\
\textsf{Das ist ein serifenloser Text}\\
\texttt{Das ist ein Schreibmaschinen Text}\\
\textnormal{Das ist ein normaler Text}

\subsubsection{Schrift Ausrichtung}
\label{sec:Schrift Ausrichtung}
\begin{quote}
Quote Text (Der gesamte Text innerhalb der Umgebung wird von beiden Seiten eingerückt)
\end{quote}
\begin{center}
Zentrierter Text (Der gesamte Text innerhalb der Umgebung wird zentriert)
\end{center}
\begin{flushleft}
Linksbündiger Text (Der gesamte Text innerhalb der Umgebung wird linksbündig)
\end{flushleft}
\begin{flushright}
Rechtsbündiger Text (Der gesamte Text innerhalb der Umgebung wird rechtsbündig)
\end{flushright}
In einer Fußnote\footnote{können zusätzliche Ergänzungen, Präzisierungen, Textverweise usw. eingeführt werden.}

\subsection{Abbildungen}

In \cref{fig:fhdw} sehen Sie das Logo der FHDW.

\begin{figure}[hbt]
\centering
\begin{minipage}[t]{.7\textwidth} % Breite, z.B. 1\textwidth		
\caption{Das Logo der FHDW} % Überschrift
\includegraphics[width=1\textwidth]{img/fhdw}\\ % Pfad
\source{Eigene Darstellung} % Quelle
\label{fig:fhdw}
\end{minipage}
\end{figure}

\subsection{Tabellen}

In \cref{tab:pin} auf Seite \pageref{tab:pin} sehen Sie die am häufigsten benutzten PINs.

\begin{table}[hbt]
\centering
\begin{minipage}[t]{.5\textwidth} % Breite, z.B. 1\textwidth		
\caption{Die am häufigsten verwendeten PINs} % Überschrift
\begin{tabularx}{\columnwidth}{rXrr}
\toprule
Rank & PIN & Percentage & Accumulated \\
\midrule
1 & 1234 & 4.34\% & 4.34\%\\
2 & 0000 & 2.57\% & 6.91\%\\
3 & 2580 & 2.32\% & 9.23\%\\
4 & 1111 & 1.60\% & 10.83\%\\
5 & 5555 & 0.87\% & 11.70\%\\
6 & 5683 & 0.70\% & 12.39\%\\
7 & 0852 & 0.60\% & 12.99\%\\
8 & 2222 & 0.56\% & 13.55\%\\
9 & 1212 & 0.49\% & 14.03\%\\
10 & 1998 & 0.43\% & 14.46\%\\
\bottomrule
\end{tabularx}
\source{Eigene Darstellung} % Quelle
\label{tab:pin}
\end{minipage}
\end{table}

\subsection{Zitate}

Ein Zitat im Fließtext ist zu sehen bei \citet{Fuller2011}.

Ein vergleichendes Zitat.\footnote{\cite[vgl.][5\psqq]{Maslennikov2011}}

Ein \enquote{wörtliches Zitat}\footnote{\cite[13\psq]{Meier2010}}

Zitat einer Quelle mit mehreren Autoren.\footnote{\cite[vgl.][32\psqq]{Hocking2011a}}


\subsection{Abkürzungen}
Bei der ersten Verwendung werden Abkürzungen ausgeschrieben: \gls{AES}.
Später wird dann automatisch nur noch die Kurzform benutzt: \gls{AES}

\subsection{Glossar}
Ein \gls{Glossar} beinhaltet Begriffserklärungen. 
Wenn du kein Glossar benötigst, entferne den Eintrag aus der Thesis.tex, damit die dazugehörige Seite nicht angezeigt wird.

\subsection{Listen}
\label{sec:Listen}
Eine einfache List mit Punkten:

\begin{compactitem}
	\item Punkt 1
	\item Punkt 2
	\item Punkt 3
\end{compactitem}

Eine einfache Liste mit Nummern:
\begin{compactenum}
	\item Punkt 1
	\item Punkt 2 
	\item Punkt 3
\end{compactenum}

Eine einfache Liste mit römischen Nummern:
\begin{compactenum}[I.]
	\item Punkt 1
	\item Punkt 2
	\item Punkt 3
\end{compactenum}

Eine einfache Liste mit Buchstaben:
\begin{compactenum}[(a)]
	\item Punkt 1
	\item Punkt 2 
	\item Punkt 3
\end{compactenum}

\subsection{Quelltext}

Listing~\ref{list:android} auf Seite~\pageref{list:android} zeigt einigen Quelltext.

\begin{figure}[bht]
\begin{lstlisting}[caption=Scanning for Wi-Fi Access Points on Android, label=list:android]
registerReceiver(new RSSIBroadcastReceiver(), 
    new IntentFilter(WifiManager.SCAN_RESULTS_AVAILABLE_ACTION));

WifiManager wifi = getSystemService(Context.WIFI_SERVICE);
wifi.startScan();

/* not thread safe */
public class RSSIBroadcastReceiver extends BroadcastReceiver {

    public void onReceive(Context context, Intent intent) {
        WifiManager wifi = getSystemService(Context.WIFI_SERVICE);
        List<ScanResult> scanResults = wifiManager.getScanResults();

        for (ScanResult scanResult : results) {
            RSSI rssi = new RSSI();
            rssi.bssi = scanResult.BSSID;
            rssi.signalLevel = scanResult.level;
        }
    }
}
\end{lstlisting}
%\footnoterule{}
%\footnotesize{Casts have been omitted for the sake of readability}
\end{figure}